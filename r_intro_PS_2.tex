% Options for packages loaded elsewhere
\PassOptionsToPackage{unicode}{hyperref}
\PassOptionsToPackage{hyphens}{url}
%
\documentclass[
]{article}
\usepackage{amsmath,amssymb}
\usepackage{iftex}
\ifPDFTeX
  \usepackage[T1]{fontenc}
  \usepackage[utf8]{inputenc}
  \usepackage{textcomp} % provide euro and other symbols
\else % if luatex or xetex
  \usepackage{unicode-math} % this also loads fontspec
  \defaultfontfeatures{Scale=MatchLowercase}
  \defaultfontfeatures[\rmfamily]{Ligatures=TeX,Scale=1}
\fi
\usepackage{lmodern}
\ifPDFTeX\else
  % xetex/luatex font selection
\fi
% Use upquote if available, for straight quotes in verbatim environments
\IfFileExists{upquote.sty}{\usepackage{upquote}}{}
\IfFileExists{microtype.sty}{% use microtype if available
  \usepackage[]{microtype}
  \UseMicrotypeSet[protrusion]{basicmath} % disable protrusion for tt fonts
}{}
\makeatletter
\@ifundefined{KOMAClassName}{% if non-KOMA class
  \IfFileExists{parskip.sty}{%
    \usepackage{parskip}
  }{% else
    \setlength{\parindent}{0pt}
    \setlength{\parskip}{6pt plus 2pt minus 1pt}}
}{% if KOMA class
  \KOMAoptions{parskip=half}}
\makeatother
\usepackage{xcolor}
\usepackage[margin=1in]{geometry}
\usepackage{color}
\usepackage{fancyvrb}
\newcommand{\VerbBar}{|}
\newcommand{\VERB}{\Verb[commandchars=\\\{\}]}
\DefineVerbatimEnvironment{Highlighting}{Verbatim}{commandchars=\\\{\}}
% Add ',fontsize=\small' for more characters per line
\usepackage{framed}
\definecolor{shadecolor}{RGB}{248,248,248}
\newenvironment{Shaded}{\begin{snugshade}}{\end{snugshade}}
\newcommand{\AlertTok}[1]{\textcolor[rgb]{0.94,0.16,0.16}{#1}}
\newcommand{\AnnotationTok}[1]{\textcolor[rgb]{0.56,0.35,0.01}{\textbf{\textit{#1}}}}
\newcommand{\AttributeTok}[1]{\textcolor[rgb]{0.13,0.29,0.53}{#1}}
\newcommand{\BaseNTok}[1]{\textcolor[rgb]{0.00,0.00,0.81}{#1}}
\newcommand{\BuiltInTok}[1]{#1}
\newcommand{\CharTok}[1]{\textcolor[rgb]{0.31,0.60,0.02}{#1}}
\newcommand{\CommentTok}[1]{\textcolor[rgb]{0.56,0.35,0.01}{\textit{#1}}}
\newcommand{\CommentVarTok}[1]{\textcolor[rgb]{0.56,0.35,0.01}{\textbf{\textit{#1}}}}
\newcommand{\ConstantTok}[1]{\textcolor[rgb]{0.56,0.35,0.01}{#1}}
\newcommand{\ControlFlowTok}[1]{\textcolor[rgb]{0.13,0.29,0.53}{\textbf{#1}}}
\newcommand{\DataTypeTok}[1]{\textcolor[rgb]{0.13,0.29,0.53}{#1}}
\newcommand{\DecValTok}[1]{\textcolor[rgb]{0.00,0.00,0.81}{#1}}
\newcommand{\DocumentationTok}[1]{\textcolor[rgb]{0.56,0.35,0.01}{\textbf{\textit{#1}}}}
\newcommand{\ErrorTok}[1]{\textcolor[rgb]{0.64,0.00,0.00}{\textbf{#1}}}
\newcommand{\ExtensionTok}[1]{#1}
\newcommand{\FloatTok}[1]{\textcolor[rgb]{0.00,0.00,0.81}{#1}}
\newcommand{\FunctionTok}[1]{\textcolor[rgb]{0.13,0.29,0.53}{\textbf{#1}}}
\newcommand{\ImportTok}[1]{#1}
\newcommand{\InformationTok}[1]{\textcolor[rgb]{0.56,0.35,0.01}{\textbf{\textit{#1}}}}
\newcommand{\KeywordTok}[1]{\textcolor[rgb]{0.13,0.29,0.53}{\textbf{#1}}}
\newcommand{\NormalTok}[1]{#1}
\newcommand{\OperatorTok}[1]{\textcolor[rgb]{0.81,0.36,0.00}{\textbf{#1}}}
\newcommand{\OtherTok}[1]{\textcolor[rgb]{0.56,0.35,0.01}{#1}}
\newcommand{\PreprocessorTok}[1]{\textcolor[rgb]{0.56,0.35,0.01}{\textit{#1}}}
\newcommand{\RegionMarkerTok}[1]{#1}
\newcommand{\SpecialCharTok}[1]{\textcolor[rgb]{0.81,0.36,0.00}{\textbf{#1}}}
\newcommand{\SpecialStringTok}[1]{\textcolor[rgb]{0.31,0.60,0.02}{#1}}
\newcommand{\StringTok}[1]{\textcolor[rgb]{0.31,0.60,0.02}{#1}}
\newcommand{\VariableTok}[1]{\textcolor[rgb]{0.00,0.00,0.00}{#1}}
\newcommand{\VerbatimStringTok}[1]{\textcolor[rgb]{0.31,0.60,0.02}{#1}}
\newcommand{\WarningTok}[1]{\textcolor[rgb]{0.56,0.35,0.01}{\textbf{\textit{#1}}}}
\usepackage{graphicx}
\makeatletter
\def\maxwidth{\ifdim\Gin@nat@width>\linewidth\linewidth\else\Gin@nat@width\fi}
\def\maxheight{\ifdim\Gin@nat@height>\textheight\textheight\else\Gin@nat@height\fi}
\makeatother
% Scale images if necessary, so that they will not overflow the page
% margins by default, and it is still possible to overwrite the defaults
% using explicit options in \includegraphics[width, height, ...]{}
\setkeys{Gin}{width=\maxwidth,height=\maxheight,keepaspectratio}
% Set default figure placement to htbp
\makeatletter
\def\fps@figure{htbp}
\makeatother
\setlength{\emergencystretch}{3em} % prevent overfull lines
\providecommand{\tightlist}{%
  \setlength{\itemsep}{0pt}\setlength{\parskip}{0pt}}
\setcounter{secnumdepth}{-\maxdimen} % remove section numbering
\ifLuaTeX
  \usepackage{selnolig}  % disable illegal ligatures
\fi
\IfFileExists{bookmark.sty}{\usepackage{bookmark}}{\usepackage{hyperref}}
\IfFileExists{xurl.sty}{\usepackage{xurl}}{} % add URL line breaks if available
\urlstyle{same}
\hypersetup{
  pdftitle={r\_intro PS 2},
  pdfauthor={Nicholus Tint Zaw},
  hidelinks,
  pdfcreator={LaTeX via pandoc}}

\title{r\_intro PS 2}
\author{Nicholus Tint Zaw}
\date{2022-11-10}

\begin{document}
\maketitle

\hypertarget{practice-summary-statistics-with-atomic-vector}{%
\section{Practice Summary Statistics with atomic
vector}\label{practice-summary-statistics-with-atomic-vector}}

First thing first, you definitely need \texttt{tidyverse} pkg for this
exercise. Don't forgot to load it at the top of your answer \texttt{rmd}
file.

Let's go back to the previous exercise on implementation of the sampling
distribution of the means.

\begin{Shaded}
\begin{Highlighting}[]
\FunctionTok{library}\NormalTok{(tidyverse)}
\end{Highlighting}
\end{Shaded}

\begin{verbatim}
## -- Attaching core tidyverse packages ------------------------ tidyverse 2.0.0 --
## v dplyr     1.1.3     v readr     2.1.4
## v forcats   1.0.0     v stringr   1.5.0
## v ggplot2   3.4.3     v tibble    3.2.1
## v lubridate 1.9.2     v tidyr     1.3.0
## v purrr     1.0.2     
## -- Conflicts ------------------------------------------ tidyverse_conflicts() --
## x dplyr::filter() masks stats::filter()
## x dplyr::lag()    masks stats::lag()
## i Use the conflicted package (<http://conflicted.r-lib.org/>) to force all conflicts to become errors
\end{verbatim}

\begin{enumerate}
\def\labelenumi{\arabic{enumi}.}
\setcounter{enumi}{1}
\tightlist
\item
  Construct the list of 100 means values from the \texttt{Sepal.Length}
  and each mean value should construct from 10 sample sizes. (sample
  size = 10, replication 100 times)
\item
  Assigned the result (vector with 100 numbers) as vector name called
  \texttt{means\_list} and .
\item
  Construct the another vector called \texttt{test} and assigned
  \texttt{1} hundred time in that vector. Use \texttt{rep()} function
  for generation of same value for 100 times.
\item
  Then, using \texttt{cbind} function to combined those two vectors''
  \texttt{test} and \texttt{mean\_list}, and treat the result as
  \texttt{data.frame} and assigned to the object name
  \texttt{mean\_list\_1}. You can try with below demo.
\end{enumerate}

\begin{Shaded}
\begin{Highlighting}[]
\NormalTok{df }\OtherTok{\textless{}{-}}\NormalTok{ iris }
\NormalTok{means\_list }\OtherTok{\textless{}{-}} \FunctionTok{replicate}\NormalTok{(}\DecValTok{100}\NormalTok{, }\AttributeTok{expr =} \FunctionTok{mean}\NormalTok{(}\FunctionTok{sample}\NormalTok{( df}\SpecialCharTok{$}\NormalTok{Sepal.Length, }\DecValTok{10}\NormalTok{, }\AttributeTok{replace =} \ConstantTok{TRUE}\NormalTok{)))}
\NormalTok{?rep  }
\end{Highlighting}
\end{Shaded}

\begin{verbatim}
## starting httpd help server ... done
\end{verbatim}

\begin{Shaded}
\begin{Highlighting}[]
\NormalTok{test }\OtherTok{\textless{}{-}} \FunctionTok{rep}\NormalTok{(}\DecValTok{1}\NormalTok{, }\AttributeTok{times =} \DecValTok{100}\NormalTok{)}
\NormalTok{means\_list\_1 }\OtherTok{\textless{}{-}} \FunctionTok{data.frame}\NormalTok{(}\FunctionTok{cbind}\NormalTok{(test, means\_list))}
\end{Highlighting}
\end{Shaded}

Repeat the same process from numbers 1 to 4, but using different sample
size and replication number this time.

\begin{itemize}
\tightlist
\item
  for \texttt{means\_list\_2}, using a sample size 30 and replication
  time 200.
\item
  for \texttt{means\_list\_3}, using a sample size 50 and replication
  time 1,000.\\
\item
  for \texttt{means\_list\_4}, using a sample size 50 and replication
  time 3,000.\\
\item
  for \texttt{means\_list\_5}, using a sample size 50 and replication
  time 10,000.
\end{itemize}

\begin{Shaded}
\begin{Highlighting}[]
\NormalTok{means\_list\_2 }\OtherTok{\textless{}{-}} \FunctionTok{data.frame}\NormalTok{(}\FunctionTok{cbind}\NormalTok{(test,}\StringTok{"means\_list"} \OtherTok{=} \FunctionTok{replicate}\NormalTok{( }\DecValTok{200}\NormalTok{, }\AttributeTok{expr =} \FunctionTok{mean}\NormalTok{(}\FunctionTok{sample}\NormalTok{(df}\SpecialCharTok{$}\NormalTok{Sepal.Length,}\DecValTok{30}\NormalTok{, }\AttributeTok{replace =} \ConstantTok{TRUE}\NormalTok{)))))}
\NormalTok{means\_list\_3 }\OtherTok{\textless{}{-}} \FunctionTok{data.frame}\NormalTok{(}\FunctionTok{cbind}\NormalTok{(test,}\StringTok{"means\_list"} \OtherTok{=} \FunctionTok{replicate}\NormalTok{( }\DecValTok{1000}\NormalTok{, }\AttributeTok{expr =} \FunctionTok{mean}\NormalTok{(}\FunctionTok{sample}\NormalTok{(df}\SpecialCharTok{$}\NormalTok{Sepal.Length,}\DecValTok{50}\NormalTok{, }\AttributeTok{replace =} \ConstantTok{TRUE}\NormalTok{)))))}
\NormalTok{means\_list\_4 }\OtherTok{\textless{}{-}} \FunctionTok{data.frame}\NormalTok{(}\FunctionTok{cbind}\NormalTok{(test,}\StringTok{"means\_list"} \OtherTok{=} \FunctionTok{replicate}\NormalTok{( }\DecValTok{3000}\NormalTok{, }\AttributeTok{expr =} \FunctionTok{mean}\NormalTok{(}\FunctionTok{sample}\NormalTok{(df}\SpecialCharTok{$}\NormalTok{Sepal.Length,}\DecValTok{50}\NormalTok{, }\AttributeTok{replace =} \ConstantTok{TRUE}\NormalTok{)))))}
\NormalTok{means\_list\_5 }\OtherTok{\textless{}{-}} \FunctionTok{data.frame}\NormalTok{(}\FunctionTok{cbind}\NormalTok{(test,}\StringTok{"means\_list"} \OtherTok{=} \FunctionTok{replicate}\NormalTok{( }\DecValTok{10000}\NormalTok{, }\AttributeTok{expr =} \FunctionTok{mean}\NormalTok{(}\FunctionTok{sample}\NormalTok{(df}\SpecialCharTok{$}\NormalTok{Sepal.Length,}\DecValTok{50}\NormalTok{, }\AttributeTok{replace =} \ConstantTok{TRUE}\NormalTok{)))))}
\end{Highlighting}
\end{Shaded}

This time, we are going to create the list to store all those
\texttt{means\_list\_x} (where \texttt{x} \texttt{1:5}) and assigned
that list as \texttt{means\_seris}. And, perform the following function
from that list.

\begin{enumerate}
\def\labelenumi{\arabic{enumi}.}
\tightlist
\item
  calculate the mean of \texttt{means\_list\_x} (where \texttt{x}
  \texttt{1:5}) from that list (using the command related to filtering
  list from lecture 2.
\end{enumerate}

For example, if we use the following command, we can get the first
dataframe. You work is to calculate the mean value of column
\texttt{mean\_list}.

\begin{Shaded}
\begin{Highlighting}[]
\NormalTok{means\_series }\OtherTok{\textless{}{-}} \FunctionTok{list}\NormalTok{(means\_list\_1,means\_list\_2,means\_list\_3,means\_list\_4,means\_list\_5)}
\end{Highlighting}
\end{Shaded}

\begin{Shaded}
\begin{Highlighting}[]
\DocumentationTok{\#\# Mean for mean\_list\_1}
\FunctionTok{mean}\NormalTok{(means\_series[[}\DecValTok{1}\NormalTok{]][[}\DecValTok{2}\NormalTok{]])}
\end{Highlighting}
\end{Shaded}

\begin{verbatim}
## [1] 5.8731
\end{verbatim}

\begin{Shaded}
\begin{Highlighting}[]
\DocumentationTok{\#\# Mean for mean\_list\_2}
\FunctionTok{mean}\NormalTok{(means\_series[[}\DecValTok{2}\NormalTok{]][[}\DecValTok{2}\NormalTok{]])}
\end{Highlighting}
\end{Shaded}

\begin{verbatim}
## [1] 5.8687
\end{verbatim}

\begin{Shaded}
\begin{Highlighting}[]
\DocumentationTok{\#\# Mean for mean\_list\_3}
\FunctionTok{mean}\NormalTok{(means\_series[[}\DecValTok{3}\NormalTok{]][[}\DecValTok{2}\NormalTok{]])}
\end{Highlighting}
\end{Shaded}

\begin{verbatim}
## [1] 5.839646
\end{verbatim}

\begin{Shaded}
\begin{Highlighting}[]
\DocumentationTok{\#\# Mean for mean\_list\_4}
\FunctionTok{mean}\NormalTok{(means\_series[[}\DecValTok{4}\NormalTok{]][[}\DecValTok{2}\NormalTok{]])}
\end{Highlighting}
\end{Shaded}

\begin{verbatim}
## [1] 5.842723
\end{verbatim}

\begin{Shaded}
\begin{Highlighting}[]
\DocumentationTok{\#\# Mean for mean\_list\_5}
\FunctionTok{mean}\NormalTok{(means\_series[[}\DecValTok{5}\NormalTok{]][[}\DecValTok{2}\NormalTok{]])}
\end{Highlighting}
\end{Shaded}

\begin{verbatim}
## [1] 5.843685
\end{verbatim}

\begin{Shaded}
\begin{Highlighting}[]
\NormalTok{?bind\_rows}
\NormalTok{df\_means\_combined }\OtherTok{\textless{}{-}} \FunctionTok{bind\_rows}\NormalTok{(means\_list\_1,means\_list\_2,means\_list\_3,means\_list\_4,means\_list\_5)}
\FunctionTok{str}\NormalTok{(df\_means\_combined)}
\end{Highlighting}
\end{Shaded}

\begin{verbatim}
## 'data.frame':    14300 obs. of  2 variables:
##  $ test      : num  1 1 1 1 1 1 1 1 1 1 ...
##  $ means_list: num  5.47 6.01 5.81 5.45 5.9 5.92 5.34 6.32 6.02 5.63 ...
\end{verbatim}

\begin{Shaded}
\begin{Highlighting}[]
\FunctionTok{View}\NormalTok{(df\_means\_combined)}
\end{Highlighting}
\end{Shaded}

\begin{enumerate}
\def\labelenumi{\arabic{enumi}.}
\setcounter{enumi}{1}
\item
  Using \texttt{bind\_rows()} function to combined all dataset from
  \texttt{means\_seris} and assigned into object called
  \texttt{df\_means\_combined}. Please make sure that your result
  dataset should have \texttt{14300} observations and 2 variables;
  \texttt{test} and \texttt{mean\_list}.
\item
  Finally, calculate the mean value for each group of \texttt{test}
  using following example. In this exercise, we are going to use
  \texttt{group\_by} function from the \texttt{tidyverse} packge and
  \texttt{\%\textgreater{}\%} operator (pipe operator).
\end{enumerate}

\begin{Shaded}
\begin{Highlighting}[]
\FunctionTok{library}\NormalTok{(tidyverse)}

\NormalTok{iris }\SpecialCharTok{\%\textgreater{}\%}
  \FunctionTok{group\_by}\NormalTok{(Species) }\SpecialCharTok{\%\textgreater{}\%}
  \FunctionTok{summarise}\NormalTok{(}\AttributeTok{mean =} \FunctionTok{mean}\NormalTok{(Sepal.Length))}
\end{Highlighting}
\end{Shaded}

\begin{verbatim}
## # A tibble: 3 x 2
##   Species     mean
##   <fct>      <dbl>
## 1 setosa      5.01
## 2 versicolor  5.94
## 3 virginica   6.59
\end{verbatim}

\end{document}
